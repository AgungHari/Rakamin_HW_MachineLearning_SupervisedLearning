\section{Pendahuluan}

\subsection{Latar Belakang}

Youtube adalah platform berbagi video yang sangat populer di seluruh dunia. Dengan jutaan video yang diunggah setiap hari, penting bagi pembuat konten untuk memahami bagaimana cara meningkatkan visibilitas dan daya tarik video mereka. Salah satu cara untuk mencapai hal ini adalah dengan menganalisis metadata dari video yang telah diunggah sebelumnya.

Dalam tugas ini, kita akan menganalisis metadata dari video-video yang telah diunggah ke Youtube. Metadata ini mencakup berbagai informasi seperti judul, deskripsi, tag, dan jumlah penonton. Dengan menganalisis metadata ini, kita dapat mengidentifikasi pola-pola yang dapat membantu pembuat konten dalam meningkatkan performa video mereka.

\subsection{Tujuan}
Tujuan dari tugas ini adalah diantara lain untuk:

\begin{itemize}
    \item Menganalisis metadata dari video Youtube untuk memahami faktor-faktor yang mempengaruhi jumlah penonton.
    \item Mengembangkan model prediksi yang dapat digunakan untuk memperkirakan jumlah penonton berdasarkan metadata video.
    \item Menerapkan regresi linier untuk memprediksi jumlah penonton video berdasarkan fitur-fitur yang tersedia dalam metadata.
    \item Membandingkan performa model regresi linier dengan model lain yang mungkin lebih kompleks.
\end{itemize}

\subsection{Batasan Masalah atau Ruang Lingkup}
Batasan masalah dalam tugas ini mencakup:
\begin{itemize}
    \item Menggunakan dataset yang diberikan oleh rakamin.
    \item Menerapkan regresi linier sebagai metode utama untuk prediksi, meskipun model lain juga akan dieksplorasi.
\end{itemize}

\subsection{Manfaat}
Manfaat dari tugas ini adalah membuat model regresi linier yang dapat digunakan untuk memperkirakan jumlah penonton video Youtube berdasarkan metadata.

\newpage
